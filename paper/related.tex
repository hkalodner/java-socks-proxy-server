\section{Related Work}
\label{sec:related}

Our proxy rotation scheme has many of the same advantages as Tor, such as the anonymity gained from routing traffic through many proxies. While this protocol is no replacement for Tor, we feel that our proxy rotation also has a few distinct advantages over Tor. Particularly, we think our method is advantageous with respect to evading censorship, where a powerful entity (e.g. a government) is actively blocking connections to proxies.

This is for three main reasons: primarily, we believe that our system is more resistant to shutting down nodes. Ideally, there would be many more nodes available for proxy rotation than there are exit nodes in the Tor network. We justify this by claiming the barrier for entry into our system is smaller than that for a Tor exit node. We hope that the barrier of entry to our system is low enough that it could be used by everyday people running proxies out of their garage.  Additionally, our system includes financial incentives for providing service, whereas Tor requires the goodwill of the exit nodes. Both of these should lead to a larger pool of proxies for someone choosing to use our system.

Additionally, We think it would be harder to identify and block all nodes running our proxy. It is fairly easy to identify Tor exit nodes as they are all published online\cite{tor:node} but, depending on the discovery scheme being used for the proxy rotation, it could be difficult to identify all proxies available to a client. We discussed the strengths and weaknesses of our discovery mechanisms in section \ref{sec:discovery}. 

Finally, we hope that the currency incentive will cause proxies to be more reliable. While Tor nodes try to stay online, they are free and there is less incentive for a server manager to ensure that their server stays up and operational at all times.

There is also a benefit of using proxy rotation to evade web tracking. Using Tor, your IP will look the same to a website for up to 10 minutes\cite{tor:faq}. Using the proxy rotation, however, will cause a user's IP to change over a single session, making them much harder to track. This leaves cookies and browser fingerprinting as one of the only ways to still track the user over a session. Combining this service with cookie blocking and anti-fingerprinting techniques could make it impossible for a remote server to link a user's traffic.

Conversely, there are also some drawbacks of the proxy rotation when compared to Tor. The proxy rotation system is more susceptible to man in the middle attacks. There is no check on proxies before they start advertising themselves for rotation. Any given node could be an adversary, making it easier to perform man in the middle attacks. This problem could be alleviated by having users always route their traffic twice, through two different proxies in the pool. If one of the proxies is trying to perform a man in the middle attack, the user will be alerted because the data presented by both proxies will differ. A malicious adversary could  \comment{(i don't know what to say about sybil attacks or how we can alleviate the problem)}
