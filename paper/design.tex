\section{Design}
\label{sec:design}

\begin{figure*}
  \centering
  \includegraphics[width=0.95\textwidth]{microtransactions.png}
  \caption{Diagram of micropayment protocol}
  \label{fig:micropayment-protocol}
\end{figure*}


\subsection{Financial Design and Motivation}

Currently the available proxy market is made up of free proxies which are slow, with bad availability, and paid proxies which use per-month credit card payments. The lack of quality in free proxies can be a problem. It has been shown that the Tor relays have a lopsided distribution with more than expected residing in less stable IP prefixes \cite{vanbever2014anonymity}. Using credit cards to pay for proxies also has its own problems and reduces the anonymity of a user, as each proxy a user shares their identifying information with could be infiltrated by an attacker or seized by an adversary with state powers. Using month to month proxies also limits the pool size of proxies due to the practical financial limitations on how many proxies a single person could afford to be subscribed to in a single month. 

We would like to employ a third economic strategy, involving the Bitcoin micropayment channel, to bring many parties to the market, on both the client and proxy side. Using Bitcoin for payment to proxies provides significant benefits over free proxies or pay by month proxies. As opposed to free proxies, using Bitcoin micropayments could make proxies much more available. Once servers are collecting a small fee, it is in their interest to maintain a certain quality of service. Specifically, the Bitcoin fees could be configured in such a way as to charge per byte, or per request. Thus, a server would make more money if it was quick (an incentive to be fast), and would only be paid if it was functioning (an incentive to be available). When compared to credit card subscriptions, the Bitcoin payment method also has advantages. Bitcoin is a pseudonymous payment system, which allows payment to a proxy without revealing the identity of the client. 
Because of the nature of the micropayments, the relationship between a client and a server is short. This is particularly desirable for the client if the client is trying to evade oppressive censorship and the proxy constantly runs the risk of being shut down at some point before the payment period ends (in which case the client would lose out on a service they have already paid for).

\subsection{Mechanics of Micropayment Channels}

The micropayment protocol visualized in Figure \ref{fig:micropayment-protocol} consists of a combination of on chain and off chain transactions. Micropayments go through an initialization phase in which the client creates (but does not broadcast) an escrow transaction with the maximum amount of spendable money in the channel, requiring both the client's and the server's signatures to unlock. Next, the client creates a refund a refund transaction which recovers all of the money in escrow and is time locked (it will only be valid after a certain number of blocks have been mined, which can be approximated with a time as each block takes an average of 10 minutes to mine). The server then signs the refund transaction.

After receiving the signed refund transaction, the client can then safely broadcast the previously created escrow transaction on the Bitcoin network. This marks the beginning of the payment phase. The client will sign transactions paying incrementally more money from the escrow to the server, and returning less money to themselves. These transactions are all double-spends of the escrow transaction so only one can be validly placed on the block chain.

At some point the business between the client and the server comes to an end, bringing about the start of the final, resolution, phase. The most typical ending of business involves the client telling the server that they are done. At this point, the server will sign the payment provided by the client which gives the server the largest amount of money and broadcast it to the Bitcoin network.

\subsection{Advantages of the Micropayment Structure}

There are a few major advantages to using Bitcoin micropayments. First, the micropayments are low risk for both the client and the server. In the initialization phase, the client does not broadcast the escrow transaction until they have a signed refund transaction from the server. This presents the client with no risk because if the server doesn't behave or fails, the client can wait for the lock time to expire and recover all of their coins from escrow. Also, the server has no reason to not sign the refund transaction because it only spends funds that already belong to the client.

During the payment phase, neither the server nor the client are at much risk. If the server stops providing service, the client will stop sending the new transactions with increased payment. Similarly, if the client stops sending new transactions, the server can stop providing its service. At the worst, the server or the client can only be cheated out of either one unit of service (e.g. megabyte of bandwidth), or one unit of payment.

The resolution phase can start in a few different ways other than the one discussed in the previous subsection. If the server doesn't hear an updated transaction from the client for an extended period of time, it will assume the client is done and closes the connection, moving to the resolution phase. Additionally, the server will move to the resolution phase if the current time becomes too close to the unlock time for the refund transaction. If the server fails, the resolution phase consists of the client waiting to be able to post the refund transaction (under normal operation, the refund transaction is never broadcast to the block chain, because it tries to spend coins that the server will have already claimed with the final micropayment of the channel). This only exists to protect the client from an unresponsive server. An additional benefit of the micropayment structure is that it can be used to pay for arbitrarily small amounts of service (for proxy rotation, the units of service still need to be large enough that the overhead of the Bitcoin transactions doesn't dominate the data flow). This is useful as a single unit of service/payment defines the risk being taken on by the server/client.

\subsection{Drawbacks of Micropayments}

There are a few drawbacks associated with the micropayment structure. First, 
the client and server must deal with the confirmation time of transactions in Bitcoin. When the client makes the original escrow transaction, the server must wait for the transaction to be confirmed before it starts serving the client. If the server does not wait for the escrow transaction to be confirmed, the client could attempt to double spend the money in the escrow, taking the money out of the payment channel. If this were to happen, the server would not be paid for the service it had already provided (making the risk larger than one service increment). This means that the escrow needs to be set up in advance of the client using the proxy. 

Another small drawback of the micropayment structure is that even though a client will recoup their funds from escrow eventually (from a dead server), they must wait for the refund transaction to become available. Currently, the waiting period is set to 24 hours, which means that a client's money would be unavailable for this time. If a client puts a lot of money in escrow with several derelict servers, they can cause a large amount of their funds to become temporarily frozen.

\subsection{Discovery}
\label{sec:discovery}

The proxy servers will be running, each listening on their respective port and IP addresses. Clients need a means of locating these proxy servers. We developed two methods for a client to discover proxies. The first method involves using a BitTorrent tracker. A BitTorrent tracker is typically used by BitTorrent clients in order to find peers to download files from. A torrent client will connect to a tracker and send an announce request. The request will include a hash of the files the client is seeking. The tracker will then send an announce response to the client with a list of peers who also have that file, based on the info hash given. In our BitTorrent discovery implementation, we created a random ``magic number'' hash that all clients and proxies would announce to a bittorrent tracker. This way the tracker's announce responses will always include the IP addresses and ports of proxies.

The second method we employed is the Bitcoin block chain itself. The Bitcoin block chain stores a record of all transactions in the Bitcoin system. Typical transactions describe where the Bitcoins are coming from and where they are going, but some transaction types can be filled with 40 bytes of data. Forty bytes of data is sufficient to store an IP address and port number. It would be impractical to interpret every 40 bytes of data on the block chain as an IP address and port (as the 40 bytes could be unrelated data). In order to detect which bytes can be interpreted as IP addresses and ports, proxy hosts will prefix the bytes of the IP address and port they are advertising with a magic number of sufficient length such that it is unlikely that the magic number appears in the normal operation of Bitcoin. The cost for a proxy to announce via Bitcoin is the transaction fee, plus a minuscule amount of Bitcoin, as the Bitcoin network doesn't accept transfers of zero Bitcoin.

\subsubsection{Comparison of Discovery Mechanisms}
The advantages of using a tracker based system is that the protocol is much simpler and doesn't require spending Bitcoins to advertise. An announcement to a BitTorrent tracker takes time on the order of seconds, whereas it takes an average of 10 minutes for a Bitcoin block to be mined. This causes a larger delay for proxies who are trying to bootstrap themselves. Thus, using trackers require less time and money for a proxy to get bootstrap. Another advantage of trackers is that there can be many trackers, as any existing tracker can now become a means of proxy discovery, or people could even run their own trackers on a smaller scale. Censors would have a harder time finding and eliminating all proxies or trackers.

On the other hand, just as it may be difficult for censors to find trackers, it may be hard for individuals to find trackers. The Bitcoin block chain provides an eventually consistent but decentralized means of storing and viewing information, meaning information on the block chain is hard to block but easy to find. To block the block chain discovery it would likely be necessary to block the Bitcoin protocol, which could be circumvented with format transforming encryption. Another option would be for censors to parse the announce messages on the block chain themselves, and censor all the addresses they find, but this could be abused by advertising arbitrary services, which would essentially force the censor to use a white-list.


\begin{figure*}
  \centering
  \includegraphics[width=1.05\textwidth]{proxydiagram.png}
  \caption{Diagram of proxy rotation}
  \label{fig:proxy-diagram}
\end{figure*}

